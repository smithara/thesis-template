\chapter{Something more serious}

A simple equation on one line is just
\begin{equation}
    A = \sin{2\pi nx} + \cos{m\phi}. \label{eqn:sincos}
\end{equation}
Maxwell's equations can be written as
\begin{align}
\nabla \cdot \bm{E} &= \frac \rho {\epsilon_0},  \label{eqn:maxwell1}\\
\nabla \cdot \bm{B} &= 0,  \label{eqn:maxwell2}\\
\nabla \times \bm{E} &= -\frac{\partial \bm{B}}{\partial t}, \label{eqn:maxwell3}\\
\nabla \times \bm{B} &= \mu_0 \left( \bm{j} + \epsilon_0 \frac{\partial \bm{E}}{\partial t}\right). \label{eqn:maxwell4}
\end{align}

Now we can reference equations such as Equation \ref{eqn:maxwell2}, and sections like Section \ref{sec:stuff}. We can cite works in the bibliography using the citekey set in their Mendeley entries, either within text like \cite{Gauss1838} or in parentheses \citep[e.g.][]{Gauss1838}.